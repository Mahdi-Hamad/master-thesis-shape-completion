\chapter{Shape Representation}
\label{ch:appendix-shape-representation}

To complement the rough notion of watertight meshes as closed surfaces with
clearly defined interior and exterior, we provide the corresponding mathematical
background for clarification.

\section{Watertight Meshes}

We follow
\cite[Section~1.3]{BotschKobbelt:2010}, \cite{Edelsbrunner:2003}
and \cite[Chapter~3]{Giblin:2010} to first introduce the necessary
terminology.

% TODO illustrations
\begin{definition}
  \begin{enumerate}[(i)]
    \item A self intersection is an intersection of two faces of the same mesh.
    \vskip -4px
    \item A non-manifold edge has more than two incident faces.
    \vskip -4px
    \item The star of a vertex is the union of all its incident faces.
    \vskip -4px
    \item A non-manifold vertex is a vertex where the corresponding star is not
    connected when removing the vertex.
    \vskip -4px
    \item A mesh is 2-manifold if it does contain neither self intersections, nor
    non-manifold edges, nor non-manifold vertices.
  \end{enumerate}
\end{definition}

% TODO cite
Illustrations of these somewhat abstract definitions can be found in
\cite[Figure~1.6]{BotschKobbelt:2010}. In general, 2-manifold meshes are
preferable to arbitrary meshes as many
algorithms and applications are not applicable to non-manifold meshes \cite{BotschKobbelt:2010}.
In our case, however, the definition of 2-manifold meshes is only motivated by
the need to formally define watertight meshes (which are sometimes also referred
to as closed meshes).
Intuitively, the only constraint missing from 2-manifold meshes is a notion
of ``closedness'', \ie a clear interior and exterior.
This becomes apparent when considering the definition
of a non-manifold edge -- which also allows edges with only one incident faces,
so-called boundary edges. 

\begin{definition}
  A 2-manifold mesh is called watertight if each edge has exactly two incident
  faces, \ie no boundary edges exist.
\end{definition}

The above definitions, while appearing abstract, are also useful in practice.
In software, \eg in MeshLab\footnote{
  \url{http://www.meshlab.net/}.
}, it is easy to identify and label non-manifold vertices, edges as well as
boundary edges to help design and work with triangular meshes.
